\documentclass[11pt, english]{article}
\usepackage{graphicx}
\usepackage[colorlinks=true, linkcolor=blue]{hyperref}
\usepackage[english]{babel}
\selectlanguage{english}
\usepackage[utf8]{inputenc}
\usepackage[svgnames]{xcolor}



\usepackage{listings}
\usepackage{afterpage}
\pagestyle{plain}

\definecolor{dkgreen}{rgb}{0,0.6,0}
\definecolor{gray}{rgb}{0.5,0.5,0.5}
\definecolor{mauve}{rgb}{0.58,0,0.82}

%\lstset{language=R,
%    basicstyle=\small\ttfamily,
%   stringstyle=\color{DarkGreen},
%    otherkeywords={0,1,2,3,4,5,6,7,8,9},
%    morekeywords={TRUE,FALSE},
%    deletekeywords={data,frame,length,as,character},
%    keywordstyle=\color{blue},
%    commentstyle=\color{DarkGreen},
%}

\lstset{frame=tb,
language=R,
aboveskip=3mm,
belowskip=3mm,
showstringspaces=false,
columns=flexible,
numbers=none,
keywordstyle=\color{blue},
numberstyle=\tiny\color{gray},
commentstyle=\color{dkgreen},
stringstyle=\color{mauve},
breaklines=true,
breakatwhitespace=true,
tabsize=3
}

\usepackage{here}


\textheight=21cm
\textwidth=17cm
%\topmargin=-1cm
\oddsidemargin=0cm
\parindent=0mm
\pagestyle{plain}

%%%%%%%%%%%%%%%%%%%%%%%%%%
% La siguiente instrucción pone el curso automáticamente%
%%%%%%%%%%%%%%%%%%%%%%%%%%

\usepackage{color}
\usepackage{ragged2e}

\global\let\date\relax
\newcounter{unomenos}
\setcounter{unomenos}{\number\year}
\addtocounter{unomenos}{-1}
\stepcounter{unomenos}
\gdef\@date{ Curso  2018 / \arabic{unomenos}}

\begin{document}

\begin{titlepage}

\begin{center}
\vspace*{-1in}
\begin{figure}[htb]
\begin{center}
\includegraphics[width=10cm]{../res/pics/logo.jpg}
\end{center}
\end{figure}

\vspace*{0.4in}
\begin{large}
\textsc{Procesadores de Lenguaje}:\\
\end{large}
\vspace*{0.2in}
\begin{Large}
\textbf{\textsc{Nombre de nuestro lenguaje}} \\
\end{Large}
\vspace*{0.3in}
\begin{large}
\@date\\
\end{large}
\vspace*{0.3in}
\rule{80mm}{0.1mm}\\
\vspace*{0.1in}
\begin{large}
Realizado por: \\

Medina Medina, David Alberto  \\
Brito Ramos, Christian  \\
Hernández Delgado, Christopher \\
López González, Néstor \\
\vspace*{0.3in}
\end{large}
\includegraphics[width=3cm]{../res/pics/LogoEscuela.jpg}
\end{center}
\end{titlepage}

\newcommand{\CC}{C\nolinebreak\hspace{-.05em}\raisebox{.4ex}{\tiny\bf +}\nolinebreak\hspace{-.10em}\raisebox{.4ex}{\tiny\bf +}}
\def\CC{{C\nolinebreak[4]\hspace{-.05em}\raisebox{.4ex}{\tiny\bf ++}}}

\tableofcontents
\newpage

\section{Definición del lenguaje (Autor: Quien termine antes)}
Introducir breve introducción del lenguaje que planteamos.
\newpage

\subsection{Tipos de datos (David)}
Aquí va el texto. Poner siempre un código de ejemplo.
\newpage

\subsection{Palabras reservadas (Christian)}
Aquí va el texto. Poner siempre un código de ejemplo.
\begin{itemize}
	\item \textbf{'continue'} . La sentencia de continue es de tipo de control de bucles. Dentro de la iteracion en un bucle, de cualquiera de los tipos (while, do-while, for), el uso de esta sentencia rompe la iteracion de dicho bucle. Provocando que se ejecute la siguiente iteracion de dicho bucle, ignorando las sentencias posteriores a "continue".
	\item \textbf{'break'} . Dentro de la iteracion en un bucle, de cualquiera de los tipos (while, do-while, for), el uso de esta sentencia rompe la iteracion de dicho bucle.
	\item \textbf{'return'} . Palabra empleada para retornar el resultado de un método o función, además de interrumpir la ejecución del mismo.
	\item \textbf{'null'} . La palabra reservada “null” indica que una variable que referencia a un objeto dentro de un array, se encuentra “sin objeto”, es decir, la variable ha sido declarada pero no apunta a ningún objeto.
	\item \textbf{'fun'} . REDACTAR.
\end{itemize}
\newpage

\subsection{Comentarios (Christian)}
Aquí va el texto. Poner siempre un código de ejemplo.
\begin{itemize}
	\item \textbf{'Comentario de línea'} . Para indicar un comentario de línea, basta con comenzar con '..', siendo el resultado final '.. {TEXTO}'.
\begin{lstlisting}[frame=single]
int a = 7
.. Variable que almacena una suma.
int suma = 0
suma = 30 + a
\end{lstlisting}
	\item \textbf{'Comentario de bloque'} . Los comentarios de varias líneas se establecen utilizando ',.' y '.,'  quedando como resultado ',. {VARIAS LÍNEAS DE TEXTO} .,'.
\begin{lstlisting}[frame=single]
int a = 7
int suma = 0
suma = 30 + a
,.
En este punto, la variable suma toma el
valor de 37.
.,
\end{lstlisting}
\end{itemize}

\newpage

\subsection{Tipos de operadores}
Aquí va el texto. Poner siempre un código de ejemplo.

\subsubsection{Operadores aritméticos (David)}
Aquí va el texto. Poner siempre un código de ejemplo.

\subsubsection{Operadores lógicos (Néstor)}
Aquí va el texto. Poner siempre un código de ejemplo.

\subsubsection{Operadores bit a bit (Christian)}
Aquí va el texto. Poner siempre un código de ejemplo.
QUE TIPOS SE APLICAN ESTOS OPERADORES?
\newline
NO ME DEJA PONER ASPERSAN
\newline
LOS OPERADORES MENOR QUE Y MAYOR QUE NO SE VISUALIZAN CORRECTAMENTE
 
\begin{itemize}
	\item \textbf{'AND} .  {EXPR}  aspersan {EXPR}.
	\item \textbf{'OR'} . {EXPR} | {EXPR}
	\item \textbf{'XOR'} . {EXPR} * {EXPR}
	\item \textbf{'DESPLAZAMIENTO DERECHA'} . {EXPR} >> {EXPR}
	\item \textbf{'DESPLAZAMIENTO IZQUIERDA'} . {EXPR} << {EXPR}

\end{itemize}

\subsubsection{Operadores de array (Christopher)}
Aquí va el texto. Poner siempre un código de ejemplo.
\newpage

\subsection{Estructuras de control}
Aquí va el texto. Poner siempre un código de ejemplo.

\subsubsection{Sentencias \texttt{if-ifelse-else} (Néstor)}
Aquí va el texto. Poner siempre un código de ejemplo.

\subsubsection{Bucle \texttt{for-forelse-else} (Christopher)}
Aquí va el texto. Poner siempre un código de ejemplo.

\subsubsection{Bucle \texttt{while-whileelse-else} (Christian)}
Aquí va el texto. Poner siempre un código de ejemplo.
\newpage

\subsection{Funciones (David)}
Aquí va el texto. Poner siempre un código de ejemplo.
\newpage

\subsection{Funciones primitivas (Néstor)}
Aquí va el texto. Poner siempre un código de ejemplo.
\newpage

\subsection{Código ejemplo (Christopher)}
Aquí va el código de ejemplo con el que probaremos nuestro compilador.

\end{document}